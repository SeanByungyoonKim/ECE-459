\documentclass[12pt]{article}

\usepackage[letterpaper, hmargin=0.75in, vmargin=0.75in]{geometry}
\usepackage{float}
\usepackage{listings}

\pagestyle{empty}

\title{ECE 459: Programming for Performance\\Assignment 3}
\author{Sean Byungyoon Kim}
\date{\today}

% Code listing style
\lstset{frame=single}

\begin{document}

\maketitle

\section*{Part 1: Crack it to Me} 

To complete this part of the assignment I generated 7500 permutations of possible secrets and sent that over to the GPU to determine if any of those secrets are valid. It stores boolean values that represents whether or not any of the permutations were valid secrets into an array, and that array is searched to determine if a valid secret exists from that collection. It then outputs the secret. 
\\ 
For a secret length of 4 using OpenCL, it took 1.406 seconds to run, compared to 0.758 seconds to run for OpenMP. Both programs were run on ecetesla0. The main difficulty that I had with the assignment was getting the kernel functions to work with different types of memory (global, local, private) as well as the lack of documentation regarding OpenCL in terms of what is possible, what isn't possible. 

\section*{Part 2: Coulomb's Law Problem} 

To complete this part of the assignment, I sent the number of particles in the input file to the GPU for calculations. 3 kernels were created and run consecutively after one another, a kernel to calculate k0 (calc\_k0), a kernel to calculate y1 (calc\_y1) and a kernel to calculate k1 and z1 (simulation). The simulation kernel also determines if the calculated z1 positions are acceptable and stores it into an array. That array is then checked to determine if all of the positions calculated were acceptable. 
\\ 
For 50 particles, it took 3.344 seconds using OpenMP while it took 0.557 seconds using OpenCL. Both programs were run using ecetesla0. The difficult part of this assignment was separating the three tasks to be calculated within the GPU, as well as making test cases for myself to test the functionality of the code. 

% \begin{lstlisting}
% #include <iostream>

% int main(int argc, char *argv[])
% {
%     std::cout << "Example code listing" << std::endl;
% }
% \end{lstlisting}

\end{document}
