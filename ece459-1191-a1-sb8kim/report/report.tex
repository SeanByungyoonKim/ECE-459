\documentclass[12pt]{article}

\usepackage[letterpaper, hmargin=0.75in, vmargin=0.75in]{geometry}
\usepackage{float}
\usepackage{url}

% Fill in these values to make your life easier
\newcommand{\iterations}{???}
\newcommand{\physicalcores}{4}
\newcommand{\virtualcpus}{8}

\pagestyle{empty}

\title{ECE 459: Programming for Performance\\Assignment 1}
\author{Sean Byungyoon Kim}
\date{January 28, 2019}

\begin{document}

\maketitle

\section*{Part 0: Resource Leak}

The resource leak was caused by the write\_png\_file function, where the write\_struct that was created was not being destroyed when it was finished being used. I fixed it by destroying the struct at the end of the function by using png\_destroy\_write\_struct. \\
\\
Another issue occured where still reachable blocks were found by valgrind within the curl library. The curl\_easy\_init was not being cleaned up properly with curl\_easy\_cleanup. To fix this issue, curl\_global\_init and curl\_global\_cleanup were used. 

\section*{Part 1: Pthreads}

My code is thread-safe because of the use of mutexes that locks resources that are shared between threads when they are being used. The threads share 3 resources between them: output\_buffer, received\_fragments and received\_all\_fragments. All other variables/resources that the threads use are localized within each individual thread so function calls involving those variables can be considered to be thread/safe. There are no race conditions within the code as they are handled by the mutexes that lock/unlock shared resources accordingly. I ran experiments on an Intel i7-4720HQ CPU @ 2.60 GHz. It has \physicalcores{} physical cores and \virtualcpus{} virtual
CPUs. Tables~\ref{tbl_sequential}~and~\ref{tbl_parallel} present my results.

\begin{table}[H]
  \centering
  \begin{tabular}{lr}
    & {\bf Time (s)} \\
    \hline
    Run 1 & 62.189 \\
    Run 2 & 59.052 \\
    Run 3 & 24.792 \\
    \hline
    Average & 48.678 \\
  \end{tabular}
  \caption{\label{tbl_sequential}Sequential executions terminate in a mean of 48.678 seconds.}
\end{table}

\begin{table}[H]
  \centering
  \begin{tabular}{lrr}
    & {\bf N=4, Time (s)} & {\bf N=64, Time (s)} \\
    \hline
    Run 1 & 20.585 & 32.248 \\
    Run 2 & 56.865 & 46.553 \\
    Run 3 & 12.521 & 27.496 \\
    \hline
    Average & 29.990 & 35.432 \\
  \end{tabular}
  \caption{\label{tbl_parallel}Parallel executions terminate in a mean of 32.711 seconds.}
\end{table}

\section*{Part 2: Nonblocking I/O}

Table~\ref{tbl_nbio} presents results from my non-blocking I/O implementation. I started $64$ requests
simultaneously.

\begin{table}[H]
  \centering
  \begin{tabular}{lr}
    & {\bf Time (s)} \\
    \hline
    Run 1 & 24.208 \\
    Run 2 & 30.011 \\
    Run 3 & 28.255 \\
    Run 4 & 27.106 \\
    Run 5 & 38.908 \\
    Run 6 & 41.815 \\
    \hline
    Average & 31.717 \\
  \end{tabular}
  \caption{\label{tbl_nbio}Non-blocking I/O executions terminate in a mean of $31.717$ seconds.}
\end{table}

\paragraph{Discussion.} At 64 requests, the non-blocking I/O implementation ran the fastest when compared to the parallel and sequential executions, although the runtimes between the non-blocking and parallel implementations was fairly small. This result is expected as the parallel and non-blocking implementation should see similar performance results and still be faster than the sequential implementation. 

\section*{Part 3: Amdahl's Law and Gustafson's Law}
I measured the user runtime of the entire program (with 4 threads) and the user runtime of the parallel portion of the program by using time ./paster\_parallel and the clock() function respectively. Than I subtracted the time taken for the parallel portion from the user total runtime. Over 3 runs, it gave an average of 2.47 seconds or the sequential portion of paster\_parallel. \\
\\
However, Amdahl's Law does not apply to paster\_parallel because it breaks the assumption that the runtime of the program can be measured accurately. Since the program receives PNG fragments randomly, the runtime can wildly vary as it'll take different amounts of time for the program to receive all the fragments it needs to build the PNG. As well, server inconsistencies can act as overhead that negatively affects the performance of the program which breaks the assumption of Amdahl's Law that overhead doesn't matter. Therefore, Amdahl's Law does not apply to paster\_parallel. \\
\\
Gustafson's Law does not apply to paster\_parallel either because of the random nature that the program receives the fragments from the server. Similar to Amdahl's Law, it becomes difficult to measure the runtime of the program when the program receives fragments randomly whiel also potentially experiencing server errors, 
\end{document}
